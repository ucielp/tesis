\chapter{Introducción} 

\section{Silenciamiento génico y ARN pequeños}


\section{miARNs en plantas}

Los miARNs son generados a partir de loci endógenos, tanto en animales como en plantas. 
controlan una gran variedad de procesos biológicos, como el desarrollo, la diferenciación ploriferación y respuesta a estrés \citep{Voinnet2009669,pmid25118717,citeulike:8816489,pmid12869753,Axtell2008}

Hasta hoy, en \textit{Aabidopsis thaliana} se han identificado más de 300 \citep{Kozomara2014} miARNs.
Se han utilizado distintos enfoques para identificar los miARNs: el clonado directo de ARN pequeños, mediante secuenciación de alto rendimiento y mediante estudios genéticos y mediante predicciones bioinformáticas \citep{citeulike:8816489}, siendo está última la más común para la mayoría de las especies.

Los miARNs en plantas están codificados por familias de genes de 1 a 32 miembros que dan lugar a miARNs maduros idénticos o muy similares.
Cada \textit{locus} perteneciente a una familia codifica un miARN maduro idéntico o casi idéntico.
Hasta el momento han sido definidas unas 42 familias de miARNs en plantas, las que regulan una amplia variedad de procesos biológicos.
Doce de dichas familias tienen como blanco ARN mensajeros que codifican factores de transcripción involucrados en el desarrollo, mientras que otras están relacionadas con rutas de respuesta a señales ambientales y hormonales, entre otros, estando la mayoría de ellas conservadas entre mono y dicotiledóneas \citep{Jones-Rhoades2006}.
Muchos de estos pequeños ARNs han aparecido recientemente en la evolución y por lo tanto aparecen en un número pequeño de especies \citep{Axtell2008,Axtell2008343}. Además está claro si tienen algún rol biológico \citep{Axtell2008343,citeulike:8816489}.

Sin embargo, existen 22 familias de miARNs que están altamente conservadas en las plantas, estando presentes en angiospermas, gimnospermas y algunas de ellas aún en plantas basales como los musgos \citep{Axtell2008,Arazi2005,pmid16623887} (ver Tabla \ref{table:table_consensus}).
Estos últimos miARNs cumplen funciones esenciales para la biología de las plantas \citep{Jones-Rhoades2006}.


\section{Biogénesis de miARNs}

Los miARNs se diferencian de otros ARNs pequeños por su particular biogénesis que implica su escisión de un precursor con extensa estructura secundaria localizado en un largo transcripto primario.
En general, la biogénesis de estos ARN pequeños comienza con la transcripción por la ARN polimerasa II \citep{Xie2005a} a partir de unidades transcripcionales propias distribuidas en el genoma \citep{Reinhart2002}.
Los transcriptos primarios, llamados pri-miARNs, pueden tener varias kilobases de longitud y sufrir modificaciones post-transcripcionales como ser splicing, capping y poliadenilación. 
Estos transcriptos contienen precursores para miARNs con extensa estructura secundaria en forma de tallo y burbuja (stem-loop) \citep{Jones-Rhoades2006}.

En animales, el procesamiento comienza en el núcleo por DROSHA y finaliza en el citoplasma por la acción de DICER.
En plantas, los precursores son procesados completamente en el núcleo a través de la acción de una ribonucleasa llamada DCL1 \citep{Reinhart2002,pmid12417148} (del inglés DICER LIKE 1) en asociación con el cofactor proteico de unión a ARN de doble hebra HYL1 \citep{Han2004} (del inglés HYPONASTIC LEAVES 1) y la proteína SERRATE \citep{Lobbes2006}.

Al parecer es la estructura secundaria por sobre la secuencia primaria del precursor la más importante en la determinación del correcto procesamiento del mismo \citep{Bologna11112012} .
El producto generado a partir de los cortes llevados a cabo por DCL1, es un dúplex miARN-miARN* que luego continúa siendo procesado por otros componentes enzimáticos hasta dar lugar al miARN maduro de 21 nt.
El paso final de la biogénesis de los miARN es la incorporación asimétrica, a partir del dúplex miARN-miARN*, del miARN maduro dentro de un complejo de silenciamiento
Este complejo se denomina RISC (del inglés RNAi Silencing Complex).
El componente central de todos los complejos de silenciamiento es un miembro de la familia de proteínas ARGONAUTA (AGO).
En Arabidopsis existen distintas proteínas AGO que participan en diferentes procesos biológicos \citep{Cellulaire2008} y la incorporación de los ARN pequeños en los distintos complejos depende de la identidad del nucleótido del extremo 5’ y de la vía de biogénesis \citep{pmid18342361,Montgomery2008,Takeda2008}. 
En la mayoría de los miARNs el nucléotido extremo 5' es una U y en general la principal efectora de la actividad es AGO1 \citep{Voinnet2009669,pmid18342361,Vazquez2004a}.
Complejos RISC similares se encuentran presentes en células animales.
Más recientemente han sido identificadas proteínas adicionales que regularían la actividad de la maquinaria de procesamiento \citep{Bologna11112012}.

En animales, los miARNs reconocen principalmente a la región 3’ no codificante de ARN mensajeros blanco inhibiendo su traducción.
En plantas es más común que los miARNs se unan a secuencias complementarias en los ARNm blanco en la región codificante señalandolos para su degradación \citep{Jones-Rhoades2006}.
En cualquier caso, es el miARN el que proporciona la especificidad contra las moléculas de ARN blanco \citep{Bartel2004}.

\section{}


\subsection{Mechanism of miRNA-Mediated mRNA Decay in Plants}

En animales existe un gran número de genes blanco medidado por miARNs y un ARNm puede estar regulado por varios miARNs.
En cambio los miARNs en plantas regulan un número limitado de genes blanco que además tienen un único sitio que es altamente complementario al miARN \citep{Voinnet2009669}.

El ARN pequeño guía al complejo RISC hacia una molécula de ARNm complementario. 
Luego del reconocimiento de ARNm blanco por complementariedad de bases, la proteína AGO1 del complejo RISC introduce un corte en un enlace fosfodiester del ARNm.
Este corte ocurre entre las posiciones 10 y 11 desde el extremo 5' del miARN, independientemente de la longitud del miARN \citep{Mallory2004,Llave2002,pmid12931144,Xie2003,pmid15057819}

Luego del corte medidado por el miARN, los fragmentos 3' son degradados  mediante la actividad de la enzima citoplasmática 5’-3’EXORIBONUCLEASA4 (XRN4) en En \textit{A. thaliana}  \citep{pmid15260969}
Los fragmentos 5' también pueden ser degradados por un complejo denominado Exosoma, el cual está encargado de diferentes funciones de degradación y procesamiento de ARNs \citep{pmid18160042}.
En Arabidopsis la degradación del fragmento 5', puede ser acelerada por uridilación en el extremo 3' por la enzima una enzima denomina HESO1 \citep{Ren2014}.

\subsection{miRNA-Mediated Translational Repression in Plants}

Los miARNs en animales, en general, son parcialmente complementarios a uno o más sitios presentes en la región 3’ no traducida de los ARNm blancos \citep{pmid12869753,pmid8252621,Fabian} y raramente sufren el tipo de corte antes mencionado. 
Además la limitada complementariedad de secuencia, permite que los miARNs de animales regulen generalmente la expresión de muchos genes blanco.
Un mecanismo que involucra la inhibición de la traducción del ARNm blanco por el miARN explica la represión de la expresión de los blancos de miARNs en animales \citep{Fabian}.
En otras ocasiones, los miARNs de animales disminuyen la vida media de los transcriptos a los que se unen \citep{pmid20703300}.
En algunos pocos miARN de plantas también se ha demostrado la existencia de un mecanismo de represión traduccional, además del corte del transcripto \citep{Schwab2005517,pmid19531599,pmid18392778,pmid18483398,pmid12893888,pmid14555699}.

