\setcounter{chapter}{8}
\chapter*{Conclusiones} 
\addcontentsline{toc}{chapter}{Conclusiones}
\setcounter{figure}{0}
\setcounter{table}{0}
\setcounter{section}{0}


\graphicspath{{Chapter2/Figs/}}


\section{Aplicaciones bioinformáticas para el estudio de interacciones miARN-gen blanco}

En cuanto a la primera parte de la Tesis y mediante diferentes estrategias y estudios, hemos alcanzado las siguientes conclusiones.

Diseñamos una estrategia para identificar genes blanco regulados por miARNs en plantas, basado en la conservación evolutiva del par miARN-gen blanco.
El enfoque requiere que la interacción miARN-gen blanco, pueda ocurrir en el contexto de un conjunto mínimo de parámetros que interactúan en diferentes especies. Pero la secuencia del gen blanco en sí, no necesariamente tiene que estar conservada.
Además, nuestro enfoque permite ajustar el número de especies requeridas como un filtro para realizar la búsqueda con diferentes sensibilidades y relaciones señal/ruido.
Utilizando esta estrategia identificamos y validamos experimentalmente nuevos genes blanco en \textit{A. thaliana}, a pesar de que este sistema ya había sido estudiado en detalles en distintos enfoques genómicos a gran escala (\citep{Allen2005207,JonesRhoades2004787,Addo-quaye2009a,German2008,Rajagopalan2006,Schwab2005517}).
Tres de los nuevos genes blanco validados tienen bulges. Parámetros empíricos usualmente le otorgan una gran penalidad a ellos, que puede llegar a ser el doble que un mismatch regular \citep{JonesRhoades2004787},  sin embargo es probable que genes blancos con bulges asimétricos sean más frecuente de lo que se pensaba previamente en plantas.
El enfoque ofrece una estrategia alternativa a otras predicciones que se basan en parámetros empíricos del par miARN-gen blanco \citep{Allen2005207,JonesRhoades2004787,citeulike:8816489,Fahlgren_chapter}.
Una ventaja de la estrategia presentada es que la interacciones miARN-gen blanco conservadas probablemente participen en procesos biológicos relevantes.
Además, esta estrategia puede ser fácilmente modificada para incorporar datos de otras bibliotecas, y/o para realizar la búsqueda de genes blanco presentes en un grupo específico de especies de plantas.

Por último, en esta parte del trabajo de Tesis, desarrollamos una herramienta web denominada comTAR para predecir potenciales genes blanco regulados por miARNs en plantas. 
Esta herramienta, es intuitiva y flexible, y puede ser utilizada para buscar familias de potenciales genes blanco de miARNs en plantas e incluso para predecir potenciales genes blanco para nuevo ARN pequeños.

\section{Estudios genómicos sobre la biogénesis de miARNs en plantas}

En la segunda parte de esta Tesis, desarrollamos un enfoque genómico a gran escala para estudiar el procesamiento de miARNs en plantas y determinamos, de esta manera, el mecanismo de procesamiento de la mayoría de los miARNs de \textit{A. thaliana} conservados evolutivamente \citep{Bologna2013}.
Encontramos que los miARNs en plantas pueden ser procesados por cuatro mecanismos, dependientes de la dirección secuencial de la maquinaria de procesamiento y del número de cortes requeridos para liberar el miARN (Figura \ref{fig:mecanismos}) \citep{Bologna2013}.
Pudimos observar que precursores procesados en el mismo mecanismo comparten determinantes estructurales, explicando el gran rango de tamaño y forma observado en precursores de miARNs en plantas \citep{Bologna2013}.

Además, desarrollamos una estrategia bioinformática y luego una herramienta intuitiva para estudiar a los intermediarios de procesamiento en plantas silvestres y en plantas mutantes en proteínas de procesamiento.
Esta herramienta puede ser utilizada para determinar la dirección de procesamiento de precursores de miARNs y además puede ser utilizada para observar diferencias en los cortes determinados por DCL1 y precisión de la maquinaria en mutantes de procesamiento.

También, realizamos un enfoque bioinformático para el estudio de la evolución y biogénesis de miARNs en plantas.
Para esto, diseñamos una implementación gráfica para la visualización integral de precursores de miARNs en distintas especies.
Utilizando estos gŕaficos estudiamos caracterizamos la evolución de precursores de miARNs en plantas con distintos mecanismos de procesamiento.
Además, mostramos que este enfoque puede ser utilizado también para estudiar precursores de miARNs en animales.
