% ************************** Thesis Abstract *****************************
% Use `abstract' as an option in the document class to print only the titlepage and the abstract.
\begin{abstract}
Los microARNs (o miARNs) son ARN no codificantes que regulan la expresión génica en animales y plantas y están implicados en procesos biológicos muy variables, como el desarrollo, la diferenciación y el metabolismo.
Con un largo de aproximadamente 21 nucleótidos, los miARNs reconocen secuencias parcialmente complementarias en los ARNm blanco, provocando su corte o arresto de la traducción.
Los miARNs han saltado rápidamente a la primera plana del interés de la comunidad científica como un nuevo nivel en el control de la expresión génica en eucariotas.
Estudios recientes han puesto de manifiesto que los miARNs están estrechamente involucrados en distintas enfermedades de importancia.
Los cálculos actuales consideran que cerca del 40\% de los genes de humanos se encuentran regulados por miARNs. 

Está generalmente aceptado que los miARNs en plantas tienen una extensiva complementariedad con sus genes blanco y su predicción por lo general se basa en el uso de parámetros empíricos deducidos de interacciones conocidos del par miARN-gen blanco. 
En este trabajo, primero desarrollamos una estrategia para la identificación de genes blanco regulados por miARNs en plantas, basado en la conservación evolutiva del par miARN-gen blanco.
Además, pudimos encontrar genes blanco específicos de Solanaceae y demostrar que la estrategia se puede utilizar para la búsqueda de genes blanco pertenecientes a un grupo determinado de especies.

A partir de estos resultados, desarrollamos una herramienta bioinformática para identificar genes blanco de miARNs basada principalmente en la conservación durante la evolución de la interacción del par miARN-gen blanco en distintas especies.
Esta herramienta fue usada para predecir nuevas interacciones y validar experimentalmente genes blanco no conocidos anteriormente en \textit{Arabidopsis thaliana}.
Algunos de ellos podrían participar de las mismas vías que genes blanco conocidos anteriormente, sugiriendo que algunos miARNs pueden controlar diferentes aspectos de un proceso biológico.

La biogénesis de los miARNs es un proceso clave porque determina la secuencia exacta de nucleótidos del ARN pequeño funcional.
Poco se sabía sobre el reconocimiento de los precursores de plantas por la maquinaria de procesamiento.
En la segunda parte de este trabajo presentamos una estrategia para estudiar aspectos mecanísticos de la biogénesis de los miARNs en plantas.
Tratamos de dilucidar la dirección de procesamiento en precursores de miARNs en \textit{Arabidopsis thaliana} a partir de los patrones de evolución de los precursores.

\end{abstract}
