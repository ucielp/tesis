% ************************** Thesis Abstract *****************************
% Use `abstract' as an option in the document class to print only the titlepage and the abstract.
\renewcommand{\chaptername}{Abstract}

\begin{abstract}
Los microARNs (o miARNs) son ARN no codificantes que regulan la expresión génica en animales y plantas, y están implicados en procesos biológicos muy variables, como el desarrollo, la diferenciación y el metabolismo.
Con un largo de aproximadamente 21 nucleótidos, los miARNs reconocen secuencias parcialmente complementarias en los ARNm blanco, provocando su corte o arresto de la traducción.
Los miARNs han saltado rápidamente a la primera plana del interés de la comunidad científica como un nuevo nivel en el control de la expresión génica en eucariotas.
Estudios recientes han puesto de manifiesto que los miARNs están implicados en distintas patologías de seres humanos.
Los cálculos actuales consideran que cerca del 40\% de los genes de humanos se encuentran regulados por miARNs. 

Está generalmente aceptado que los miARNs en plantas tienen una extensiva complementariedad con sus genes blanco y su predicción por lo general se basa en el uso de parámetros empíricos deducidos de interacciones conocidas y validadas experimentalmente.
En este trabajo, primero desarrollamos una estrategia para la identificación de genes blanco regulados por miARNs en plantas, basado en la conservación evolutiva del par miARN-gen blanco.
Además, pudimos encontrar genes blanco específicos de Solanaceae y demostrar que la estrategia se puede utilizar para la búsqueda de genes blanco pertenecientes a un grupo determinado de especies.
Esta estrategia fue usada para predecir nuevas interacciones no conocidas anteriormente en \textit{Arabidopsis thaliana}, que luego fueron validados experimentalmente.
Algunos de los nuevos genes identificados podrían participar de las mismas vías que genes blanco conocidos anteriormente, sugiriendo que algunos miARNs pueden controlar diferentes aspectos de un proceso biológico.

A partir de estos resultados, desarrollamos una herramienta bioinformática disponible en un servidor de acceso público para identificar genes blanco de miARNs basada principalmente en la conservación durante la evolución de la interacción del par miARN-gen blanco en distintas especies.
La herramienta también brinda una descripción de varios parámetros de las interacciones miARN-gen blanco en distintas especies, que puede ser útil para mejorar los sistemas de predicción y describir la co-evolucion de los miARNs y los genes blanco. 

La biogénesis de los miARNs es un proceso clave porque determina la secuencia exacta de nucleótidos del ARN pequeño funcional, que luego determina los genes a ser regulados. 
Los precursores de miARNs de plantas son muy variables en forma y tamaño en comparación con los precursores de miARNs de plantas.
Y por lo tanto, poco se sabía sobre el reconocimiento de los precursores por la maquinaria de procesamiento.
En la segunda parte de este trabajo presentamos una estrategia para estudiar la conservación de los precursores de miARNs de plantas en distintas especies identificando regiones y dominios presentes en distintas especies.
A partir de estos resultados, desarrollamos un herramienta de visualización para poder analizar fácilmente la conservación de la estructura primaria y secundaria de los precursores de miARNs.
El análisis las mismas reveló que existe una marcada correlación entre la conservación de los precursores y el mecanismo de procesamiento.
Los resultados muestran patrones de conservación en los dominios necesarios para el procesamiento, y permiten deducir un modelo general del reconocimiento del ARN durante la biogénesis de miARNs.

\end{abstract}
