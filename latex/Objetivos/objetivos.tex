\setcounter{chapter}{2}
\chapter*{Objetivos} 
\addcontentsline{toc}{chapter}{Objetivos}
\setcounter{figure}{0}
\setcounter{section}{0}

\section{Objetivo general} 

Uno de los objetivos general de este trabajo de Tesis consiste en identificar a los genes regulados por miARNs y descubrir sus roles en plantas.
Además, como objetivo queremos contribuir al conocimiento de la regulación del procesamiento de los miARNs en plantas.
Se espera que los resultados de esta Tesis sirvan no solo para alcanzar los objetivos de investigación planteados sino también para promover el desarrollo de la Bioinformática como una disciplina que brinda una oportunidad única para que, a partir de investigaciones en las ciencias básicas, pueda hallarse el camino hacia el desarrollo de aplicaciones de interés estratégico para el país.

\section{Objetivos específicos}

\begin{enumerate}
    \item Identificar genes regulados por miARNs en plantas.
    \begin{itemize}
        \item Diseñar una estrategia para la identificación de genes blanco regulados por miARNs en plantas, basado en la conservación evolutiva del par miARN-gen blanco.
        \item Desarrollar una herramienta web para la predicción de genes blanco de miARNs en diferentes especies de plantas.		
    \end{itemize}
    \item Estudiar la biogénesis de los miARNs en plantas.
    \begin{itemize}
		\item Identificación y caracterización de precursores de miARNs con mecanismos de procesamiento distintos.
		\item Identificación de intermediarios de procesamiento en plantas mutantes en proteínas de procesamiento.
        \item Caracterizar la relación entre la evolución de los precursores de miARNs en plantas y los mecanismos de procesamiento determinados previamente.
    \end{itemize}
\end{enumerate}
