\setcounter{chapter}{2}
\chapter*{Objetivos} 
\addcontentsline{toc}{chapter}{Objetivos}
\setcounter{figure}{0}
\setcounter{table}{0}
\setcounter{section}{0}

\section{Objetivo general} 

Las interacciones ARN-ARN son esenciales durante la regulación de la expresión génica por miARNs, y en los procesos de silenciamiento génico en general.
La identificación de los genes blanco por miARNs se basa en la interacción de dos moléculas de ARN donde la interacción no es necesariamente perfecta. 
En principio, se puede hipotetizar que cierto número de bases desapareadas en determinadas posiciones permitirán la actividad del ARN pequeño en el contexto de la interacción miARN-gen blanco. 

Por otro lado, la secuencia exacta del miARN depende del procesamiento de precursores más largos con extensa estructura secundaria.
Este proceso es preciso en plantas, pese a que los precursores de miARNs son variables en secuencia, forma y tamaño.

El primer objetivo de esta Tesis se basa en diseñar estrategias para identificar genes regulados en miARN en plantas, tratando de seleccionar aquellas interacciones que tengan una mayor probabilidad de tener un impacto significativo en la biología de las plantas.
En este proceso hemos elegido una combinación de herramientas bioinformáticas, incluyendo la conservación de la interacción miARN-gen blanco en distintas especies.

Un segundo objetivo busca encontrar patrones comunes y dominios de ARN que sirvan de guía durante el proceso de biogénesis de miARNs.
Aquí también hemos elegido una combinación de elementos incluyendo el análisis de experimentos de secuenciación de alto rendimiento y el análisis de la conservación evolutiva de precursores de miARNs.

Se espera que los resultados de esta Tesis sirvan no solo para alcanzar los objetivos de investigación planteados sino también para promover el desarrollo de la Bioinformática como una disciplina que brinda una oportunidad única para que, a partir de investigaciones en las ciencias básicas, pueda hallarse el camino hacia el desarrollo de aplicaciones de interés estratégico para el país.

\section{Objetivos específicos}

\begin{enumerate}
    \item Identificar genes regulados por miARNs en plantas.
    \begin{itemize}
        \item Diseñar una estrategia para la identificación de genes blanco regulados por miARNs en plantas, basado en la conservación evolutiva del par miARN-gen blanco.
        \item Desarrollar una herramienta web para la predicción de genes blanco de miARNs en diferentes especies de plantas.		
    \end{itemize}
    \item Estudiar la biogénesis de los miARNs en plantas.
    \begin{itemize}
        \item Desarrollar herramientas que permitan el análisis de los intermediarios de procesamiento de miARNs en plantas a partir de bibliotecas de secuenciación masiva de ARN.
		\item Identificar y caracterizar de precursores de miARNs en distintas especies que tengan mecanismos de procesamiento distintos.
        \item Caracterizar la relación entre la evolución de los precursores de miARNs en plantas y los mecanismos de procesamiento determinados previamente.
    \end{itemize}
\end{enumerate}
